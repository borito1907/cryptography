\documentclass{article}
\usepackage{geometry}
\usepackage{graphicx} % Required for inserting images
\usepackage{amsmath, amsthm, amssymb}
\usepackage{mathtools}
\DeclarePairedDelimiter\ceil{\lceil}{\rceil}
\DeclarePairedDelimiter\floor{\lfloor}{\rfloor}
\usepackage{parskip}
\newgeometry{vmargin={15mm}, hmargin={24mm,34mm}}
\theoremstyle{definition} 

\newtheorem{definition}{Definition}
\newtheorem{theorem}{Theorem}[section]
\newtheorem{lemma}[theorem]{Lemma}
\newtheorem{corollary}{Corollary}[theorem]
\newtheorem{proposition}[theorem]{Proposition}
\newcommand{\N}{\mathbb{N}}
\newcommand{\Z}{\mathbb{Z}}
\newcommand{\R}{\mathbb{R}}
\newcommand{\Q}{\mathbb{Q}}
\newcommand{\xtoc}{\lvert x \rvert ^{c}}
\newcommand{\bitstring}[1]{\{0,1\}^{#1}}
\newcommand{\Gen}{\textbf{Gen}}
\newcommand{\Sample}{\textbf{Sample}}
\newcommand{\Eval}{\textbf{Eval}}
\newcommand{\Invert}{\textbf{Invert}}

\DeclareMathOperator*{\E}{\mathbb{E}}

\title{Cryptography}
\date{March 2024}
\author{Boran Erol}

\begin{document}

\maketitle

The Goldreich-Levin Theorem proves existence of probabilistic learning algorithms
for linear functions.



\newpage

\section{Presentation in Katz and Lindell}



\begin{theorem}
    Assume OWFs exist. Then, there's a one-way function $g$ and a hard-core predicate
    $hc$ of $g$.
\end{theorem}

7.3.2 in Katz and Lindell

Notes about Claim 7.15 in Katz and Lindell

For $x \in S_{n}$, we're being careful with the size of the set and relaxing the success probability.
For $x \notin S_{n}$, we're being careful with the success probability and relaxing the set size.

Notes about Claim 7.16 in Katz and Lindell

Union bound the failure probabilities for $r$ and $r \oplus e_{i}$.

Finish the proof by using the Chernoff bound.

\end{document}


