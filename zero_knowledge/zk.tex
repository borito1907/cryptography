\documentclass{article}
\usepackage{geometry}
\usepackage{graphicx} % Required for inserting images
\usepackage{amsmath, amsthm, amssymb}
\usepackage{mathtools}
\DeclarePairedDelimiter\ceil{\lceil}{\rceil}
\DeclarePairedDelimiter\floor{\lfloor}{\rfloor}
\usepackage{parskip}
\newgeometry{vmargin={15mm}, hmargin={24mm,34mm}}
\theoremstyle{definition} 

\newtheorem{definition}{Definition}
\newtheorem{theorem}{Theorem}[section]
\newtheorem{lemma}[theorem]{Lemma}
\newtheorem{corollary}{Corollary}[theorem]
\newtheorem{proposition}[theorem]{Proposition}
\newcommand{\N}{\mathbb{N}}
\newcommand{\Z}{\mathbb{Z}}
\newcommand{\R}{\mathbb{R}}
\newcommand{\Q}{\mathbb{Q}}
\newcommand{\xtoc}{\lvert x \rvert ^{c}}
\newcommand{\bitstring}[1]{\{0,1\}^{#1}}
\newcommand{\Gen}{\textbf{Gen}}
\newcommand{\Sample}{\textbf{Sample}}
\newcommand{\Eval}{\textbf{Eval}}
\newcommand{\Invert}{\textbf{Invert}}

\DeclareMathOperator*{\E}{\mathbb{E}}

\title{Zero-Knowledge}
\date{March 2024}
\author{Boran Erol}

\begin{document}

\maketitle

\section{Zero-Knowledge Proofs}

The interaction reveals nothing beyond the validity of the proposition.

The definition avoids "what is knowledge" altogether!

Here's an example of ZK using color blindness:

ZK is useful for identification, upgrading semi-honest security to malicious security, 



ZK is at the right level of abstraction. It's simple enough to be studied and realized. The feasibility/limitations
delinate what is attainable.

ZK is just a means to an end. There are weaker definitions that are also useful.

If you want efficient protocols that use ZK, you need to understand the inner workings of ZK. If you use
ZK in a modular fashion, you'll get cool protocols, but they won't have optimal efficiency.

\subsection{Definitions and Motivation}



ZK proofs for all of NP is Avi Wigderson's favorite result. Here's how he explains it.
Let's say you are arguing with your friend. You'd like to convince him that
you are correct. However, your friend doesn't trust you. Therefore, in order to
convince your friend, you have to provide information. ZK proofs are convincing without
providing any knowledge!

There's a problem with ZK protocol, even 3 decades after its founding: we still don't have a
good, concise language to express our proofs. We have machines interacting, noticeable, negligible probabilities,
it's just all a mess. One big open question in protocol design is to come up with a easy to use language.

\begin{definition}[Zero Knowledge]
    
\end{definition}

\begin{definition}[Statistical ZK]
    $\forall PPT V^{*}: \exists PPT S: \forall x \in L: \forall z$

    \[ S(x,z) \approx_{s} View_{V^{*}}(P(x,w),V^{*}(x,z)) \]
\end{definition}

\begin{lemma}
    If NP $\subseteq$ SZK, the polynomial hierarchy collapses to the second level.
\end{lemma}
\begin{proof}
    
\end{proof}

A natural follow-up question to ask is for which relaxations of SZK can we still have
proofs for all languages in NP?

There are two possible paths to take:

\begin{enumerate}
    \item Make the ZK condition computational -- \textbf{CZK Proofs}. In this situation, you have
    undisputable soundness but the secrecy is computational.
    \item Make the soundness condition computational using WI -- \textbf{SZK arguments}. In this situation,
    you always have some lingering doubts about the soundness but the secrecy is unconditional.
\end{enumerate}

\subsection{SZK argument for Graph Hamiltonicity}

\begin{theorem}
    If statistically-hiding commitments exist, then there's a SZK argument for Graph
    Hamiltonicity.
\end{theorem}
\begin{proof}
    We'll use the exact same protocol we used to construct a PZK protocol
    for HAM.
\end{proof}

However, the soundness of this protocol is not negligible. Whether this protocol is still
SZK under parallel repetition is an open problem. To construct a SZK argument for every language
in NP, we can instead use WI.

Whenever you have a relation for an NP language, this relation is fully determined
by the verifier from the NP definition. However, notice that $L \in$ NP can have infinitely
many different NP-Relations $R_{L}$.

\subsection{ZK for QR}

Let $x$ be a QR mod $N$. Assume $P$ has some $w$ such that $x = w^{2} \pmod{N}$.

\begin{enumerate}
    \item $P$ samples $r$ uniformly at random from $Z^{*}_{\N}$ and sends $y = r^{2}$.
    \item $V$ samples $b$ uniformly at random and sends it to $P$.
    \item If $b = 0$, the prover sends $z = r$. $V$ checks if $z^{2} = y$.
    \item If $b = 1$, the prover sends $z = wr$. $V$ checks if $z^{2} = xy$.
\end{enumerate}

\begin{proof}[soundness]
    Assume $x$ is not a quadratic residue. Then, if you send $y$ that is a QR, there's no
    way to satisfy the condition at $b = 1$. If you send $y$ that is not a QR, you can't satisfy
    $b = 0$.
\end{proof}

\subsection{Questions and Answers}

Is there a public-coin ZK im(possibility) result?

\newpage 

\section{Proof of Knowledge}

\subsection{Definition and Motivation}

Proving that something exists versus proving that you know something is not the same thing. Proof
of knowledge tries to ensure that the prover actually knows the witness. Throughout these definitions,
$n = \lvert x \rvert$. 

\begin{definition}
    A proof system has \textbf{knowledge soundness if error $\kappa$} if there exists
    a PPT $\kappa$ such that for every prover $P^{*}$ that convinces $V$ of $x$ with 
    probability $\epsilon > \kappa$, $K^{P^{*}(\cdot)}$ outputs $w$ such that
    $(x,w) \in R_{L}$ with probability at least $\epsilon(n) - \kappa(n)$.
\end{definition}

Here's an equivalent definition:

\begin{definition}
    A proof system has \textbf{knowledge soundness if error $\kappa$} if there exists
    a $\kappa$ such that for every prover $P^{*}$ that convinces $V$ of $x$ with 
    probability $\epsilon > \kappa$, $K^{P^{*}(\cdot)}$ outputs $w$ such that
    $(x,w) \in R_{L}$ and runs in time $\frac{n}{\epsilon(n) - \kappa(n)}$
\end{definition}

Here's a stronger definition:

\begin{definition}[strong proof of knowledge]
    A proof system has \textbf{strong knowledge soundness} if there's a negligible function $\mu$
    and a PPT $K$ such that for every prover $P^{*}$ that convinces $V$ of $x$ with 
    probability $\epsilon > \mu$, $K^{P^{*}(\cdot)}$ outputs $w$ such that
    $(x,w) \in R_{L}$ with probability at least $1 - \mu(n)$.
\end{definition}

\begin{lemma}
    Knowledge soundness implies soundness.
\end{lemma}

Notice that knowledge soundness is a property of the verifier.

\begin{lemma}
    Sequential composition reduces knowledge error exponentially.
\end{lemma}

In fact, in PoK, we can replace exponentially small error with zero error.

\begin{lemma}
    Let $(P,V)$ be an interactive protocol with exponentially small knowledge soundness error.
    Then, there's an interactive protocol $(P^{\prime}, V^{\prime})$ with zero knowledge
    soundness error.
\end{lemma}
\begin{proof}
    The idea is to run the knowledge extractor in parallel with a brute-force search
    on the witness.


\end{proof}

\subsection{Constructing ZKPOKs}

Consider the regular ZK protocol for QR.

Just like simulations, we're going to rewind the prover using knowledge extractors. This makes sense:
if we could extract the knowledge without rewinding, the proof would no longer be ZK!

\begin{proof}[proof of knowledge soundness]
    We assume WLOG that $P^{*}$ has its randomness hardwired and is
    deterministic. Let $\kappa = \frac{1}{2}$.

    Here's the knowledge extractor:
    \begin{enumerate}
        \item $K$ invokes $P^{*}$ and receives some $y$.
        \item $K$ sends $P^{*}$ the query $b = 0$ and receives $z_{0}$.
        \item $K$ rewinds and sends $P^{*}$ the query $b = 1$ and receives $z_{1}$.
        \item $K$ outputs $w = z_{1}/z_{0} \pmod{N}$.
    \end{enumerate}

    If $P^{*}$ convinces $V$ with probability more than $\kappa$, since it is deterministic,
    it answers both queries truthfully. We thus conclude the proof.
\end{proof}

Let's now consider the regular ZK protocol for HAM.

\begin{proof}[proof of knowledge soundness]
    We assume WLOG that $P^{*}$ has its randomness hardwired and is
    deterministic. Let $\kappa = \frac{1}{2}$.

    Here's the knowledge extractor:
    \begin{enumerate}
        \item $K$ invokes $P^{*}$ and receives a commitment $c$.
        \item $K$ sends $P^{*}$ the query $b = 0$ and receives a cycle with $w$.
        \item $K$ rewinds and sends $P^{*}$ the query $b = 1$ and receives $\pi, \tilde{G}$.
        \item $K$ outputs the cycle in the graph using \textbf{obvious, but finish still.}
    \end{enumerate}

    If $P^{*}$ convinces $V$ with probability more than $\kappa$, since it is deterministic,
    it answers both queries truthfully. We thus conclude the proof.
\end{proof}

These examples might lead you to think that proving ZKPOK is easy, but these examples are
only because the probability analysis is very clean. In general, with arbitrary probabilities,
proving ZKPOK is extremely annoying.

Sigma protocols try to solve this issue.

The protocols we have constructed so far have high error. Let's now reduce the
error to negligible using sequential composition.

Let $n = \lvert x \rvert$. 

\begin{theorem}
    Running HAM $n$ times sequentially is strong ZKPOK.
\end{theorem}
\begin{proof}
    Consider the following extractor strategy.
\end{proof}

In fact, it can be proven that it's impossible to get a strong ZKPOK with negligible error
without sequential composition.

\subsection{Applications of ZKPOK}

In practice, the alternative defintion of ZKPOK has an issue. How do we know when
$K$ is expected polynomial time? Depending on $\epsilon(n)$, $K$ might run in
expected exponential time.

A classic use of ZKPOK is the following:

Within a protocol, the prover proves the proof. To prove security, the simulator needs the witness, so
it needs to run the knowledge extractor.

Usually, when using ZKPOKs in proofs of security, the simulator plays the role of the verifier
and interacts with the prover. If the verifier rejects, the simulator just halts. If the
verifier accepts, then the simulator has to extract the witness.

What is the expected running time of the this simulator?

The probability that $P$ convinces $V$ is $\epsilon(n)$. Notice that we have no clue
what $\epsilon(n)$ is. For simplicity, let's assume $\kappa(n) = 0$. Then, the expected
running time of the simulator is 

\[ (1 - \epsilon(n)) \cdot poly(n) + \epsilon(n) \cdot \frac{poly(n)}{\epsilon(n)} = poly(n)\]

Let's now assume that $\kappa(n)$ is negligible. Then,

\[ (1 - \epsilon(n)) \cdot poly(n) + \epsilon(n) \cdot \frac{poly(n)}{\epsilon(n) - \kappa(n)} = poly(n) + \frac{\epsilon(n)}{\epsilon(n) - \kappa(n)}\]

Notice that this is not necessarily negligible!

\textbf{Witness-extended emulation}

\begin{lemma}
    If $(P,v)$ is a ZKPOK, then there exists a witness extended emulator for $(P,V)$.
\end{lemma}

In MPC, we can also formalize an ideal ZK functionality:

\[ \mathcal{F}_{zk}((x,w),x) = (\lambda, R(x,w))\]

\begin{lemma}
    If $(P,V)$ is a ZKPOK, then it securely computes the ideal ZK functionality.
\end{lemma}

\subsubsection{Applications of ZKPOK}

\begin{enumerate}
    \item A ZK proof for QNR
    \item Non-Oblivious Encryption
    \item Prove a property of statistically commmited value.
    \item Identification Schemes
\end{enumerate}

\subsubsection{A ZK proof for QNR}

\begin{enumerate}
    \item $V$ samples $b \xleftarrow{\$} \{0,1\}$, $y \xleftarrow{\$} \Z_{N}$.
    \item 
\end{enumerate}

POK allows you to get implicit statements and turn it into explicit knowledge.

\subsubsection{Non-Oblivious Encryption}

Provide an encryption and prove that you know what's encrypted.

Motivation:
\begin{enumerate}
    \item Prevent copying (e.g., in auction).
    \item Guarantee non-malleability (didn't take a previous ciphertext and maul).
\end{enumerate}

\subsubsection{Prove Property of Statistical Commited Value}

Consider a statistically-hiding commitment scheme. Because hiding is statistical, a commitment value
$c$ can be a commitment to any message.

Suppose the you're committing to your identity. In this scenario, it's often useful to prove a property
of your identity (the range of your age). Here, since existence is guaranteed, ZK proofs are useless.
Instead, the committer can prove that they know a decommitment to a message with a certain property.

\newpage

\section{Constant-Round CZK Proofs for NP}

Goldreich and Krawczyk proved that languages with constant-round, public-coin BBZK protocols are in BPP. In other words,
constant-round, public-coin BBZK protocols are useless.

We'd like to construct a CZK proof with negligible soundness and constant round complexity. Recall that parallel repetition
achieves this, but constructing a simulator for parallel repetition is an open problem.

We'll need to address:
\begin{enumerate}
    \item Malleability 
    \item Aborts in simulation
\end{enumerate}

We'll require statistically hiding commitments for the verifier and statistically binding
commitments for the prover. Essentially, because the prover is unbounded, we're ensuring security
against the prover in both cases.

There's a technical issue:

$V^{*}$ might decide to decommit depending on the commitment it receives.

Here's a naive attempt to construct a simulator:

\begin{enumerate}
    \item Commit to garbage values.
    \item If $V^{*}(c) = ABORT$, halt.
    \item If $V^{*}(c) \neq ABORT$, rewind and adjust the garbage based on the decommitment received. Repeat this process until $V^{*}$ doesn't abort.
\end{enumerate}

Let

\[ s(n) = \Pr[\text{$V^{*}$ doesn't abort given garbage commitments}]\]

\[ t(n) = \Pr[\text{$V^{*} $doesn't abort given valid commitments}]\]

Notice that the expected number of repetition for the simulator in the third step is $s(n)/t(n)$. By the computational hiding of the 
commitment scheme, the difference between $s(n)$ and $t(n)$ is negligible. However, this isn't enough! If $s(n) = 2^{-n}$ and 
$t(n) = 2^{-2n}$, we still get exponentially many repetitions. 

In 1996, Goldreich and Kahan show how to get around this issue. At a high level, they fix this problem by estimating the probability
that $V^{*}$ aborts. The analysis is complicated. Alon Rosen thinks its overkill for such a simple protocol.

In 2004, Rosen provided an alternative solution with a much simpler analysis at the cost of adding two rounds.
The idea of the protocol is to make $V^{*}$ commit in a way that allows the simulator to extract the value of
$b$ before $c$ is even sent.

In order to achieve this, the verifier creates a secret share of $b$ and sends the secret shares to $V^{*}$ before
$V^{*}$ commits. The simulator can rewind the verifier to learn $b$ before it commits to $c$ by learning both shares.
In the real protocol, $V^{*}$ learns nothing by the security of the secret-sharing scheme.

Notice that this doesn't seem to be a POK. The extractor can't rewind and change its $b$, since it has committed to it. Lindell,
in 2013, solves this issue in "A note on the constant-round ZKPOKs". The analysis of this protocol is even more complicated than
GK96.

\newpage

\section{Witness Indistinguishability}

WI is a relaxation of ZK for languages in NP. It was introduced by Feige and Shamir in 1990.

\begin{definition}
    We say that a protocol $\pi$ is \textbf{witness indistinguishable} if 
    $\forall V^{*}(x,z): \forall (x,z,w_{1},w_{2}):$

    \[ View_{V^{*}}(P(x,w_{1}), V^{*}(x,z)) \approx_{c} View_{V^{*}}(P(x,w_{2}), V^{*}(x,z)) \]
\end{definition}

\begin{definition}
    We say that a protocol $\pi$ is \textbf{witness independent} if 
    $\forall V^{*}(x,z): \forall (x,z,w_{1},w_{2}):$

    \[ View_{V^{*}}(P(x,w_{1}), V^{*}(x,z)) \approx_{s} View_{V^{*}}(P(x,w_{2}), V^{*}(x,z)) \]
\end{definition}

\begin{lemma}
    ZK implies WI.
\end{lemma}
\begin{proof}
    \[ View_{V^{*}}(P(x,w_{1}), V^{*}(x,z)) \approx_{c} S^{*}(x,z) \approx_{c} View_{V^{*}}(P(x,w_{2}), V^{*}(x,z)) \]
\end{proof}

\begin{corollary}
    If statistically-binding commitments exist, then every language in NP has a witness
    indistinguishable proof.
\end{corollary}

\begin{corollary}
    If statistically-hiding commitments exist, then every language in NP has a witness
    independent argument.
\end{corollary}

\begin{definition}
    We say that an interactive protocol $(P,V)$ is \textbf{unbounded simulation} 
    if $\forall PPT V^{*}: \exists S: \forall x \in L: \forall w \in R_{L}(x):\forall z$

    \[ View_{V^{*}}(P(x,w), V^{*}(x,z)) \approx_{c} S(x) \]
\end{definition}

\begin{lemma}
    $(P,V)$ has an unbounded simulation if and only if it's witness indistinguishable.
\end{lemma}
\begin{proof}
    Assume $(P,V)$ has an unbounded simulation. Then,

    \[ View_{V^{*}}(P(x,w_{1}), V^{*}(x,z)) \approx_{c} S(x) \approx_{c} View_{V^{*}}(P(x,w_{2}), V^{*}(x,z)) \]

    Conversely, ...
\end{proof}

Notice that WI is useless for unique witness NP languages. You can reveal the witness and still satisfy
the definition. Like ZK, WI is a property of the prover.

Notice that the protocol can leak information about the set of all witnesses.

If $P$ disregards the auxiliary input, WI is trivial. Therefore, for exponential-time provers, WI is irrelevant
since $P$ can just ignore the auxiliary input and brute-force search the witness. Since we require the prover
to be polynomial time, we now have \textbf{computational soundness} instead of unconditional soundness.
Because of this, these are also called arguments instead of proofs.

\begin{lemma}
    WI is closed under parallel repetition.
\end{lemma}
\begin{proof}
    Without loss of generality, we prove the theorem for executing two protocols in parallel.


\end{proof}


\subsection{Open Problems}

\begin{enumerate}
    \item Can Strong WI be achieved non-interactively?
    \item 
\end{enumerate}

\newpage

\section{Fiat-Shamir Heuristic (Transform)}



\newpage

\section{NIZKs}

\subsection{Definitions and Motivation}

ZK proofs seem to depend on 3 crucial ingredients:

\begin{enumerate}
    \item Interaction
    \item Secret Random Coins of the Verifier
    \item Computational Hardness Assumptions
\end{enumerate}

NIZKs try to remove interaction from the picture. They were
first introduced by Blum, Feldman, Micali in 1988 in the CRS
model.

\begin{definition}
    A pair of PPT algorithms $(P,V)$ is a \textbf{NIZK proof system} if 
\end{definition}

Notice that the simulator doesn't get access to the witness whereas the
prover does.

NIZKs can also be used to produce CCA2 secure PKE systems.

NIZKs seem related to constant-round ZK protocols.
However, NIZKs require a shared reference string.
If you can construct a secure-coin flipping protocol
when executed in parallel, this gives you a constant-round 
ZK proof from a NIZK. The Goldreich Kahan 1996 paper can be 
seen as constructing a secure coin-flipping protocol 
under parallel repetition.

Blum, Santis, Micali, Persiano improved on this later.

We're going to assume a common \textbf{random} string, which we're
going to denote with $crs$.

Rafi assumes that the prover is polynomial time. There are
constructions where the prover is unbounded, but Rafi doesn't care.

The protocol is going to be zero-knowledge if
there's a simulator $S$ that can produce $crs^{\prime}$
and $\Pi^{\prime}$ that is indistinguishable from 
$crs$ and $\Pi$.

Intuitively, the simulator is going to encode a trapdoor 
into $crs^{\prime}$ which it will use to convince the verifier, but
the prover won't be able do such a thing since $crs$ in the actual
interaction is fully random.

This is similar to the idea behind witness indistinguishability.

Rafi has a variation on the definition where the simulators
$crs$ with the trapdoor can be used in the original protocol
without the prover being able to crack the trapdoor.

Feige, Lapidot, Shamir (FLS)

One-way trapdoor permutations.

Let $\mathcal{F}$ be a family of OWPs.

The prover can sample $f, f^{-1}$ using its
own private randomness. Using this OWP, the prover
can slowly reveal the hardcore bits of this OWP to the
verifier.

Intuitively, the prover has some wiggle room at 
committing to certain strings, but it can't control
all the randomness it's committing to.

First of all, through magic, $P$ commits to
a cycle graph $G^{\prime}$. 

Then, $P$ sends $V$ a permutation $\Pi$ such that
the cycle in $G^{\prime}$ and the cycle in $\Pi(G)$ overlap.
$P$ then opens all non-edges of $\Pi(G)$ such that
it overlaps with the non-edges of the original graph $G^{\prime}$.

We're showing that $G$ is isomorphic to $G^{\prime}$
without telling $V$ what $G^{\prime}$ is apart from the
fact that

In general, (in 2004) it's unknown how to get adaptive
NIZK from non-adaptive NIZK. However, it is possible to
get from non-adaptive soundness to adaptive soundness.

\subsection{Adaptive Soundness from Non-Adaptive Soundness}

Let $Bad_{k} = \bar{L} \cap \bitstring{k}$.

The intuition is as follows: for length $k$ strings, repeat
the proof $2k$ times to achieve adaptive soundness.

Using this, for any given $x \in Bad_{k}$, we'll prove that
only a fraction of $2^{-2k}$ random strings allow the prover
to cheat. Then, summing over all possible bad strings and
using a union bound, only a fraction of $2^{-k + 1}$ strings
allow the prover to cheat.

\subsection{NIZK in CRS from NIZK in Hidden-Bits using Trapdoor OWPs}

In the Hidden-Bits model, the prover is given some
random string $r$ with $\lvert r \rvert = n$. The prover,
along with its proof $\pi$, sends $I \subseteq [n]$ to the verifier.
Then, the verifier gets all $r_{i}$'s with $i \in I$
and learns nothing about the other random bits.

\begin{theorem}
    Assuming the existence of trapdoor OWPs, given 
    a NIZK proof system in the hidden-bits model
    $(P^{\prime}, V^{\prime})$, we can construct a
    NIZK proof system $(P,V)$ in the CRS model.
\end{theorem}
\begin{proof}
    Let $(P^{\prime}, V^{\prime})$ a be NIZK proof system in the hidden-bits model.
    Let $k$ be the security parameter for $(P^{\prime}, V^{\prime})$.

    Let $\mathcal{F}$ be a trapdoor OWP family.

    Here are some assumptions we make:

    \begin{enumerate}
        \item The set of "legal" $(f,f^{-1})$ is efficiently decidable. In other words, given $(f,f^{-1})$, you can
        efficiently decide whether it was output by $\Gen$.
        \item For randomly chosen $x$, $h(x)$ is a uniformly random bit.
    \end{enumerate}

    Here are some simplifying assumptions without loss of generality:

    \begin{enumerate}
        \item The random string generated by $Gen(1^{k})$ has length $k$. Thus, there are at most $2^{k}$
        distinct $f$'s.
        \item The soundness error of $(P^{\prime}, V^{\prime})$ is at most $2^{-2k}$. We can ensure that this condition
        holds by running $2k$ copies of $(P^{\prime}, V^{\prime})$ in parallel. For a more substantive discussion, review the
        argument for upgrading non-adaptive soundness to adaptive soundness.
        \item 
    \end{enumerate}

    Let $n$ be the length of the random string $r^{\prime}$ given to $P^{\prime}$ in the original NIZK system.

    Let $r$ be a random string of length $nk$.

    Intuitively, $P$ will just use the hardcore bit of the trapdoor OWP $f$ to get $n$ hidden random bits.

    \begin{enumerate}
        \item 
    \end{enumerate} 

    The completeness of the new protocol is immediate.

    Let's now prove the soundness of the new protocol.
    
    Fix $(f,f^{-1})$. Then, for any uniformly random $r$, $r^{\prime}$ will also be uniformly distributed.

    Here, it is crucial that we're fixing $(f,f^{-1})$ \textbf{before} considering a random $r$. If 
    $(f,f^{-1})$ is dependent on $r$, our claim definitely doesn't hold

    Then, by the soundness of $(P^{\prime}, V^{\prime})$, we have that 

    \[ \Pr[\text{$P^{*}$ can cheat using $f$}] \leq 2^{-2k} \] 

    Recall that there are at most $2^{k}$ choices of $f$. Therefore,

    \[ \Pr[\text{$P^{*}$ can cheat using ANY $f$}] \leq 2^{k} \cdot 2^{-2k} = 2^{-k} \]
    
    Let's now prove that the new protocol is ZK. Let $Sim^{\prime}$ be the simulator for $(P^{\prime}, V^{\prime})$.

    Here's how we construct $Sim(1^{k},x)$:

    \begin{enumerate}
        \item Run $Sim^{\prime}(1^{k},x)$ to obtain $(r^{\prime}_{I}, \pi, I)$.
        \item Run $Gen(1^{k})$ to obtain $(f,f^{-1})$.
        \item If $i \in I$, set $r_{i} = f(z_{i})$ where $h(z_{i}) = r_{i}^{\prime}$ and $z_{i} \xleftarrow{\$} \bitstring{k}$.
        \item If $i \notin I$, set $r_{i} \xleftarrow{\$} \bitstring{k}$.
    \end{enumerate}


\end{proof}

\subsection{NIZK for Graph Hamiltonicity from Trapdoor OWPs in the CRS Model (Feige, Lapidot, Shamir)}

\newpage

\section{NIWIs}



\end{document}


